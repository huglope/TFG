Este capítulo se corresponde con la segunda etapa de la metodología CRISP-DM, En el se explicará la naturaleza de los datos y sus características, así como los valores atípicos que presentan y sus sesgos.

\section{Origen de los datos}  \label{sec.origen-datos}
Los datos que se han utilizado para desarrollar este trabajo, se han obtenido de  conjuntos de datos diseñados para entrenar Sistemas de Detección de Intrusión de Red (NIDS) basados en el aprendizaje automático. El dataset en cuenstión  forma parte de un análisis realizado en la Universidad de Queensland, Australia.\cite{luay2025NetFlowDatasetsV3}


El dataset utilizado es NF-UNSW-NB15-v3, este es una versión basada en NetFlow del conocido conjunto de datos UNSW-NB15, mejorada con características adicionales de NetFlow y etiquetada de acuerdo con sus respectivas categorías de ataque. 

\section{Tipos de ataques registrados en los datos} \label{sec.tipo-ataques}
En esta sección se explican los tipos de datos presentes en el Dataset que se utiliza para entrenar a los modelos del proyecto. Se explica en que consiste cada tipo de ataque registrado así como el número exacto de ataques de cada tipo presente.

El conjunto de datos consiste en un total de 2\,365\,424 flujos de datos, donde 127\,639 (5,4\%) son muestras de ataque y 2\,237\,731 (94,6\%) son benignos. Los flujos de ataque se clasifican en nueve clases, cada una representando una amenaza a la red distinta. La siguiente tabla proporciona una distribución detallada del conjunto de datos:

\begin{table}[H]
\label{tab:attacks-tab}
\begin{tabular}{|l|c|>{\RaggedRight}p{10cm}|} % Ajusta el ancho (8cm) según necesites
\hline
\rowcolor[HTML]{C0C0C0} 
\textbf{Clase} & \textbf{Cantidad} & \textbf{Descripción} \\ \hline
Benigno & 2\,237\,731 & Flujos normales no maliciosos. \\ \hline
Fuzzers & 33\,816 & Tipo de ataque en el que el atacante envía grandes cantidades de datos aleatorios que hacen que un sistema se bloquee y también apuntan a descubrir vulnerabilidades de seguridad en un sistema. \\ \hline
Analysis & 2\,381 & Un grupo que presenta una variedad de amenazas que se dirigen a aplicaciones web a través de puertos, correos electrónicos y scripts. \\ \hline
Backdoor & 1\,226 & Una técnica que tiene como objetivo eludir los mecanismos de seguridad respondiendo a aplicaciones específicas de clientes construidos. \\ \hline
DoS & 5\,980 & La denegación de servicio es un intento de sobrecargar los recursos de un sistema informático con el objetivo de evitar el acceso o la disponibilidad de sus datos. \\ \hline
Exploits & 42\,748 & Son secuencias de comandos que controlan el comportamiento de un host a través de una vulnerabilidad conocida. \\ \hline
Generic & 19\,651 & Un método que se dirige a la criptografía y causa una colisión con cada cifrado de bloques. \\ \hline
Reconnaissance & 17\,074 & Una técnica para recopilar información sobre un host de red, también se conoce como sonda. \\ \hline
Shellcode & 4\,659 & Un malware que penetra en un código para controlar el host de una víctima. \\ \hline
Worms & 158 & Ataques que se replican y se extienden a otros sistemas. \\ \hline
\end{tabular}
\centering
\caption{Clasificación de amenazas de seguridad}
\end{table}

\section{Parámetros de los datos} \label{sec.param-datos}
En esta sección se explicarán el significado de cada uno de los 55 parámetros que componen cada fila del Dataset seleccionado.

\url{https://arxiv.org/pdf/2503.04404}

\begin{itemize}
    \item \textbf{IPV4\_SRC\_ADDR}: Dirección IPv4 de origen
    \item \textbf{IPV4\_DST\_ADDR}: Dirección IPv4 de destino
    \item \textbf{L4\_SRC\_PORT}: Número de puerto de origen de la capa 4
    \item \textbf{L4\_DST\_PORT}: Número de puerto de destino de la capa 4
    \item \textbf{PROTOCOL}: Byte identificador del protocolo IP
    \item \textbf{L7\_PROTO}: Protocolo de aplicación (numérico) de la capa 7
    \item \textbf{IN\_BYTES}: Número de bytes entrantes
    \item \textbf{OUT\_BYTES}: Número de bytes salientes
    \item \textbf{IN\_PKTS}: Número de paquetes entrantes
    \item \textbf{OUT\_PKTS}: Número de paquetes salientes
    \item \textbf{FLOW\_DURATION\_MILLISECONDS}: Duración del flujo en milisegundos
    \item \textbf{TCP\_FLAGS}: Acumulativo de todos los flags TCP
    \item \textbf{CLIENT\_TCP\_FLAGS}: Acumulativo de todos los flags TCP del cliente
    \item \textbf{SERVER\_TCP\_FLAGS}: Acumulativo de todos los flags TCP del servidor
    \item \textbf{DURATION\_IN}: Duración del flujo Cliente a Servidor (mseg)
    \item \textbf{DURATION\_OUT}: Duración del flujo Cliente a Servidor (mseg)
    \item \textbf{MIN\_TTL}: TTL mínimo del flujo
    \item \textbf{MAX\_TTL}: TTL máximo del flujo
    \item \textbf{LONGEST\_FLOW\_PKT}: Paquete más largo (bytes) del flujo
    \item \textbf{SHORTEST\_FLOW\_PKT}: Paquete más corto (bytes) del flujo
    \item \textbf{MIN\_IP\_PKT\_LEN}: Longitud del paquete IP más pequeño del flujo observado
    \item \textbf{MAX\_IP\_PKT\_LEN}: Longitud del paquete IP más grande del flujo observado
    \item \textbf{SRC\_TO\_DST\_SECOND\_BYTES}: Bytes/segundo de origen a destino
    \item \textbf{DST\_TO\_SRC\_SECOND\_BYTES}: Bytes/segundo de destino a origen
    \item \textbf{RETRANSMITTED\_IN\_BYTES}: Número de bytes TCP retransmitidos del flujo (src->dst)
    \item \textbf{RETRANSMITTED\_IN\_PKTS}: Número de paquetes TCP retransmitidos del flujo (src->dst)
    \item \textbf{RETRANSMITTED\_OUT\_BYTES}: Número de bytes TCP retransmitidos del flujo (dst->src)
    \item \textbf{RETRANSMITTED\_OUT\_PKTS}: Número de paquetes TCP retransmitidos del flujo (dst->src)
    \item \textbf{SRC\_TO\_DST\_AVG\_THROUGHPUT}: Tasa de transferencia promedio de origen a destino (bps)
    \item \textbf{DST\_TO\_SRC\_AVG\_THROUGHPUT}: Tasa de transferencia promedio de destino a origen (bps)
    \item \textbf{NUM\_PKTS\_UP\_TO\_128\_BYTES}: Paquetes cuyo tamaño IP es <= 128
    \item \textbf{NUM\_PKTS\_128\_TO\_256\_BYTES}: Paquetes cuyo tamaño IP es > 128 y <= 256
    \item \textbf{NUM\_PKTS\_256\_TO\_512\_BYTES}: Paquetes cuyo tamaño IP es > 256 y <= 512
    \item \textbf{NUM\_PKTS\_512\_TO\_1024\_BYTES}: Paquetes cuyo tamaño IP es > 512 y <= 1024
    \item \textbf{NUM\_PKTS\_1024\_TO\_1514\_BYTES}: Paquetes cuyo tamaño IP es > 1024 y <= 1514
    \item \textbf{TCP\_WIN\_MAX\_IN}: Ventana TCP máxima (src->dst)
    \item \textbf{TCP\_WIN\_MAX\_OUT}: Ventana TCP máxima (dst->src)
    \item \textbf{ICMP\_TYPE}: Tipo ICMP * 256 + Código ICMP
    \item \textbf{ICMP\_IPV4\_TYPE}: Tipo ICMP
    \item \textbf{DNS\_QUERY\_ID}: ID de transacción de la consulta DNS
    \item \textbf{DNS\_QUERY\_TYPE}: Tipo de consulta DNS (ej., 1=A, 2=NS..)
    \item \textbf{DNS\_TTL\_ANSWER}: TTL del primer registro A (si existe)
    \item \textbf{FTP\_COMMAND\_RET\_CODE}: Código de retorno del comando del cliente FTP
    \item \textbf{FLOW\_START\_MILLISECONDS}: Marca de tiempo de inicio del flujo en milisegundos
    \item \textbf{FLOW\_END\_MILLISECONDS}: Marca de tiempo de fin del flujo en milisegundos
    \item \textbf{SRC\_TO\_DST\_IAT\_MIN}: IAT mínimo (src->dst)
    \item \textbf{SRC\_TO\_DST\_IAT\_MAX}: IAT máximo (src->dst)
    \item \textbf{SRC\_TO\_DST\_IAT\_AVG}: IAT promedio (src->dst)
    \item \textbf{SRC\_TO\_DST\_IAT\_STDDEV}: Desviación estándar del IAT (src->dst)
    \item \textbf{DST\_TO\_SRC\_IAT\_MIN}: IAT mínimo (dst->src)
    \item \textbf{DST\_TO\_SRC\_IAT\_MAX}: IAT máximo (dst->src)
    \item \textbf{DST\_TO\_SRC\_IAT\_AVG}: IAT promedio (dst->src)
    \item \textbf{DST\_TO\_SRC\_IAT\_STDDEV}: Desviación estándar del IAT (dst->src)
\end{itemize}

\section{Patrones preliminares, valores atípicos y sesgos} \label{sec.segos-datos}

Tras analizar los datos originales del dataset, se han encontrado características que afectarían de forma negativa al entrenamiento del modelo y por lo tanto, a su correcto funcionamiento posteriormente. A continuación, se mencionan cuales han sido las característcas problematicas de los datos encontradas.


Tras estudiar los datos, se descubre que algunos parámetros presentan valores infinitos, estos valores alteran la distribución inherente de las variables, introduciendo sesgos significativos en el proceso de aprendizaje. Los algoritmos de entrenamiento, diseñados para operar con valores numéricos finitos, pueden comportarse de manera impredecible o inestable ante la presencia de infinitos, dificultando la convergencia hacia una solución óptima. Asimismo, la interpretación de las características con valores infinitos se vuelve ambigua, comprometiendo la capacidad del modelo para establecer relaciones significativas con otras variables y para generalizar correctamente a datos futuros que no contengan tales valores extremos. En consecuencia, la presencia de infinitos puede degradar sustancialmente el rendimiento y la fiabilidad del modelo entrenado.

En un principio, puede parecer que los datos están sesgados por las direcciones IPv4 de los dispositivos origen. Esto se debe a que solo se producen ataques desde las direcciones con máscara 175.45.176.255. Tras realizar pruebas excluyendo este parámetro del entrenamiento del modelo, se ha llegado a la conclusión de que este parámetro no recibe un peso muy alto y no altera los resultados de las métricas del modelo.


\section{Preparación de los datos ¿SEPARAR EN OTRO CAPÍTULO?}\label{sec.preparacion-datos}
En esta sección se modificarán los datos que presentan patronees, valores atípicos o sesgos que se comentan en la sección anterior. \ref{sec.segos-datos}

 Para evitar que estos datos produzcan errores en la ejecución del algoritmo que entrena al modelo, se ha optado por eliminarlos.


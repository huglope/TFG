En este capítulo se trata el entendimiento del problema. Tal y como se comenta en el capítulo dos, es la fase inicial de la metología CRISP-DM. A continuación, se alinean los objetivos técnicos con las necesidades del negocio y con el problema a resolver. Se definen requisitos, se identifican métricas de éxito y se trata de dar comprensión sobre el contexto organizacional.


\section{¿Qué es un ataque a un sistema informático?}
Un ataque a un sistema informático constituye una acción deliberada y no autorizada que explota vulnerabilidades con el objetivo de comprometer la confidencialidad, integridad o disponibilidad de los datos y recursos del sistema. Esta actividad maliciosa puede manifestarse a través de diversas técnicas, incluyendo la inyección de código malicioso, la denegación de servicio, el acceso no autorizado y la ingeniería social. Su ejecución busca obtener beneficios ilícitos, interrumpir operaciones o dañar la infraestructura tecnológica.

La consecuencia de un ataque puede variar desde la pérdida o alteración de información sensible hasta la paralización completa de los servicios ofrecidos por el sistema. La identificación, análisis y mitigación de estas amenazas representan un aspecto fundamental en la seguridad informática, requiriendo la implementación de medidas preventivas y reactivas para proteger los activos digitales de una organización o individuo.

\section{Tipos de ataque a sistemas informáticos}


\section{¿Qué es TCP?}

El Protocolo de Control de Transmisión (TCP) constituye uno de los protocolos fundamentales de la capa de transporte del modelo TCP/IP, sobre el cual se sustenta gran parte de la comunicación en redes IP, incluyendo Internet. Su diseño se orienta a proporcionar un servicio de transferencia de datos fiable, ordenado y con detección de errores entre aplicaciones que se ejecutan en sistemas finales diferentes. Para lograr esta fiabilidad, TCP establece una conexión virtual punto a punto entre las aplicaciones comunicantes mediante un proceso de "three-way handshake", lo que permite la negociación de parámetros de la conexión y la sincronización de los números de secuencia iniciales.

\section{INSERTAR IMAGEN DE COMO FUNCIONA TCP}

El protocolo garantiza la entrega ordenada de la información al receptor mediante la asignación de números de secuencia a cada byte transmitido, permitiendo así la reordenación en caso de que la información no llegue al receptor en el orden correcto. La fiabilidad se logra a través de un mecanismo de acuse de recibo (acknowledgment, ACK) positivo con retransmisión, donde el receptor confirma la recepción correcta de los paquetes de información, y el emisor retransmite aquellos partes de la información para los que no recibe confirmación dentro de un tiempo límite (timeout).

\subsection{¿Qué es un segmento TCP?}
Una vez establecida la conexión, TCP divide los datos de la aplicación en unidades más pequeñas denominadas segmentos. Un segmento o paquete TCP constituye la unidad de datos fundamental que se intercambia a través de una red utilizando el mencionado protocolo TCP. Este segmento encapsula una porción de los datos de la capa de aplicación, precedida por una cabecera TCP. 

La cabecera TCP contiene información de control esencial para la funcionalidad del protocolo, incluyendo los números de puerto de origen y destino que identifican las aplicaciones comunicantes, los números de secuencia y de acuse de recibo (ACK) que garantizan la entrega ordenada y fiable, las banderas de control que indican el propósito del segmento (establecimiento de conexión, finalización, ACK, entre otros muchos), y otros campos como la ventana de recepción para el control de flujo y la suma de verificación para la detección de errores.


\section{INSERTAR IMÁGEN DE COMO LA ESTRUCTURA DE UN PAQUETE TCP}

En el proceso de transmisión, el segmento TCP se encapsula a su vez dentro de un paquete IP (Protocolo de Internet) para su enrutamiento a través de la red. El paquete IP añade su propia cabecera con las direcciones IP de origen y destino, entre otra información necesaria para el transporte a nivel de red. 

\section{INSERTAR IMÁGEN DE COMO LA ESTRUCTURA DE UN PAQUETE IP}

\section{Importancia de protegerse frente a un ataque}

La importancia de protegerse frente a ataques informáticos radica en la salvaguarda de activos digitales críticos, la garantía de la continuidad operativa y la preservación de la confianza y la reputación. En un entorno digital cada vez más interconectado, los ataques informáticos representan una amenaza significativa para individuos, organizaciones y la sociedad en su conjunto, pudiendo acarrear consecuencias devastadoras.

Para las organizaciones, las implicaciones de un ataque informático pueden ser aún más costosas. Estas implicaciones incluyen pérdidas financieras directas debido al robo de fondos, la interrupción de las operaciones comerciales, los costes de recuperación y las posibles sanciones regulatorias. Además, se puede producir un daño significativo a la reputación y la pérdida de la confianza de los clientes, lo que a largo plazo afecta la viabilidad del negocio. Los ataques también pueden resultar en el robo de propiedad intelectual, secretos comerciales e información estratégica, otorgando ventajas competitivas a adversarios. 

Por otra parte, la interrupción de servicios críticos, como energía, comunicaciones o sanidad, puede tener consecuencias graves para la sociedad en su conjunto.

La protección frente a ataques informáticos no es solo una cuestión de seguridad tecnológica, sino una necesidad imperante en la actualidad para proteger activos valiosos, asegurar la continuidad de las actividades, mantener la confianza de los usuarios y garantizar la estabilidad y el bienestar en el mundo digital actual. La implementación de prácticas de seguridad robustas y la concienciación sobre las amenazas cibernéticas son fundamentales en a la hora de defenderse de estos ataques.


\section{Importancia de detectar los ataques rápidamente}

La detección temprana de ataques informáticos constituye un pilar fundamental en la ciberseguridad moderna debido a su capacidad para mitigar consecuencias críticas. Cuando un sistema logra identificar intrusiones o actividades maliciosas en sus fases iniciales, se reducen significativamente los daños operativos y económicos. Esta rapidez de respuesta permite contener amenazas antes de que comprometan infraestructuras completas, preservando tanto la integridad de los datos como la continuidad del negocio.

Desde una perspectiva técnica, la identificación inmediata limita la superficie de ataque, impidiendo que los actores maliciosos escalen privilegios o se propaguen lateralmente por la red. En el ámbito regulatorio, cumple con los estrictos plazos que exigen normativas como el Reglamento General de Protección de Datos (RGPD), que obliga a notificar violaciones de seguridad en un máximo de 72 horas. Además, desde el punto de vista económico, reduce los costes asociados a las reparaciones, que suelen multiplicarse exponencialmente cuando los ataques permanecen indetectados durante largos períodos.


La capacidad de detectar rápidamente anomalías en el tráfico de red, accesos no autorizados o patrones de comportamiento sospechosos no solo protege los activos digitales, sino que también salvaguarda la reputación institucional. Organizaciones con sistemas de detección temprana robustos demuestran proactividad ante clientes y socios comerciales, generando confianza en su capacidad para manejar información sensible. Esta anticipación resulta especialmente crítica en entornos donde la disponibilidad del servicio es primordial, como en las infraestructuras críticas anteriormente mencionadas.

\section{Soluciones comerciales o actuales a estos problemas}


\section{Requisitos}  \label{sec.requisitos} 
Como se ha comentado en el punto \ref{sec.objetivos} \nameref{sec.objetivos}, 
el principal objetivo del proyecto es desarrollar un modelo neuronal que detecte la presencia de ataques en una red informática y los clasifique según su tipo. Para cumplir con dicho objetivo, se considera imprescindible cumplir con los requisitos que se listan a continuación.

\subsection{Requisitos Funcionales}   \label{sec.req-funcionales}
\textbf{Primera versión de requisitos, no me convencen mucho}
\begin{itemize}  
    \item \textbf{RF-1}: El sistema deberá detectar cuales de las conexiones podrían ser potenciales intrusiones en la red.
    \item \textbf{RF-2}: El sistema deberá clasificará las conexiones en 10 categorías predefinidas en \nameref{tab:attacks-tab}.  
	\item \textbf{RF-3}: El sistem deberá ser capaz de procesar formatos estándar de logs como son Syslog, NetFlow y PCAP.
	\item \textbf{RF-4}: El sistema deberá diferenciar entre ataques conocidos (basados en firmas) y desconocidos (basados en anomalías).
	\item \textbf{RF-5}: El sistema deberá ofrecer API REST para conexión con SIEMs (Splunk, IBM QRadar)
	\item \textbf{RF-6}: Generar alertas automatizadas con nivel de criticidad (bajo/medio/alto).
	\item \textbf{RF-7}: Proveer recomendaciones de mitigación básicas (ej. bloquear IPs maliciosas)

	
	
\textbf{¿Debería integrar el modelo en algún sistema o crear un script o alguna forma para comunicarme con él?}
		
\end{itemize}  

\subsection{Requisitos No Funcionales}   \label{sec.req-no-funcionales}
\begin{itemize}  
    \item \textbf{RNF-1}: Latencia <50 ms en redes de 10Gbps (requisito crítico para SOC~\cite{nist2021ai}).  
    \item \textbf{RNF-2}: Interfaz accesible para usuarios no técnicos (evaluado con test SUS~\cite{brooke1996sus}).  
\end{itemize}  

\subsection{Reglas de Negocio}   \label{sec.reglas-neogcio}
\begin{itemize}  
    \item \textbf{RB-1}: Coste operativo mensual no superará \$10,000 (aprobado por Comité de Seguridad).  
    \item \textbf{RB-2}: Alertas de ransomware requerirán confirmación humana antes de aislamiento de red.  
\end{itemize}  
\section{Contexto organizacional} \label{sec.contexto-organizacional}


\section{Objetivos del proyecto}
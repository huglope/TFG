
En este capítulo se exponen las conclusiones derivadas del desarrollo y evaluación del modelo neuronal diseñado para la clasificación de conexiones benignas y malignas. El enfoque propuesto consiste en un modelo de clasificación binaria que, una vez realiza la distinción entre conexiones benignas y malignas, redirige aquellas conexiones clasificadas como malignas hacia un modelo de clasificación multiclase. Este último modelo es capaz de diferenciar las conexiones malignas en nueve clases distintas, permitiendo una identificación más detallada y precisa de los diversos tipos de intrusiones. Los resultados obtenidos destacan la eficacia de ambos modelos, evidenciando su capacidad para abordar de manera jerárquica y especializada la clasificación. Finalmente, se presentan las limitaciones del estudio y se plantean posibles líneas de trabajo futuro.

\section{Conclusiones}

El primer objetivo del proyecto, consistente en diseñar e implementar un modelo capaz de detectar intrusiones en redes informáticas y proporcionar una clasificación preliminar de las mismas, se ha cubierto de forma satisfactoria. Este primer objetivo del proyecto se ha alcanzado al obtener un modelo de clasificación binaria capaz de distinguir entre conexiones benignas y malignas con un \textit{Recall} superior a $0.99$. Con respecto a la clasificación de intrusiones o conexiones malignas, se ha desarrollado un modelo de clasificación multiclase, que recibe las conexiones clasificadas por el modelo binario como malignas y clasifica estas conexiones en nueve clases diferentes de intrusiones en sistemas informáticos con un \textit{F1-weighted} cercano a $0.6$.

El segundo objetivo del proyecto, según el cual los modelos de detección debían ser modelos neuronales, se ha alcanzado satisfactoriamente. Este objetivo se logró mediante el diseño de modelos de clasificación binaria (MsCB) y multiclase (MsCM), ambos con una arquitectura compuesta por dos capas. La primera de las capas en todos los modelos corresponde con la capa oculta, es la capa que recibe los atributos de entrada del modelo y la que permite que se pueda resolver el problema no lineal que se propone en este trabajo. Esta primera capa cuenta con un número diferente de neuronas en función de la arquitectura del modelo que se observe. La segunda capa por su parte, es la que corresponde con la salida del modelo. Esta segunda capa cuenta con una única neurona en el caso de los MsCB y con nueve neuronas en el caso de los MsCM.  La utilización de un Perceptrón Multicapa (MLP) en ambos modelos de clasificación permite aprovechar la capacidad de las redes neuronales para aprender representaciones complejas y no lineales, mejorando la capacidad de clasificación

El tercer objetivo del proyecto, consistente en evaluar y comparar los modelos generados utilizando un conjunto de datos real y complejo, se ha alcanzado de manera adecuada. Tal y como se expone en el Capítulo \ref{cap.ent-datos}, correspondiente a la fase de entendimiento de los datos de la metodología CRISP-DM, el conjunto de datos empleado para dicha evaluación ha sido el \texttt{NF-UNSW-NB15-v3}. \cite{luay2025NetFlowDatasetsV3}. Este conjunto de datos es semi-sintético, esto se debe a que los flujos benignos son registros reales de la conexión entre dos sistemas, mientras que los flujos correspondientes a los ataques se generaron en un entorno controlado. Para evitar que los datos sintéticos de los flujos maliciosos perjudicasen el entrenamiento y la evaluación de los modelos, se transformaron los datos, eliminando aquellos atributos que presentaban valores que diferían con la realidad, como fue el caso de las direcciones IP.

Evaluar el grado en el que se han alcanzado los objetivos académicos es más complejo que evaluar si los objetivos del proyecto se han conseguido cumplir. Esta diferencia se debe a que mientras que los objetivos del proyecto son medibles cuantitativamente, los objeticos académicos no son medibles objetivamente. Sin embargo, teniendo en cuenta el trabajo desarrollado se considera que si se han alcanzado todos los objetivos académicos fijados.

El primer objetivo académico, centrado en comprender el funcionamiento de los modelos neuronales mediante \texttt{PyTorch} y las métricas de evaluación, se ha abordado a lo largo del desarrollo del proyecto. Al haber desarrollado varios modelos con difertes arquitecturas y calcular las métricas tanto de la fase de validación como de la fase de evaluación utilizando la biblioteca \texttt{PyTorch}, se puede afirmar que se ha comprendido el funcionamiendo de los modelos neuronales, alcanzando de esta manera el primer objetivo académico.

El segundo objetivo académico, consiste en aprender las características de varios de los tipo de modelos neruonales que existen, se ha alcanzado de manera satisfactoria. En la Sección \ref{subsec.tiposmodel} del Capítulo \ref{cap.modelos}, se explican algunos de los modelos neuronales que más se utilizan en la actualidad y se comentan las caracterísitcas de cada tipo de modelo, proporcionando una visión de las tareas en las que mejor desempeño muestra cada tipo. Tras analizar las características de los diferentes tipos de modelos neuronales, se determinó que el Perceptrón Multicapa (MLP) era la opción más adecuada para este problema, dado que los datos presentan una naturaleza no lineal y el MLP tiene la capacidad de aprender representaciones complejas y no lineales, lo cual resulta fundamental para resolver este tipo de clasificación. Para redactar la sección mencionada fue necesario entender cuales eran las caraterísticas de cada tipo de modelo neuronal, por lo que el segundo objetivo académico se ha cumplido correctamente durante el desarrollo de este trabajo.

El tercer objetivo académico, se centra en descubrir el potencial de las redes neuronales para optimizar y mejorar las tecnologías de la información, incluyendo la ciberseguridad de los sistemas, se ha cubiertod de manera satisfactoria . Tras el desarrollo de este trabajo se ha descubierto que los modelos neuronales tienen una gran capacidad para resolver problemas de clasificación con una complejidad elevada. Además, resulta fascinante el hecho de que los pesos de las neuronas se puedan ajustar en cualquier momento para converger hacia una solución más generalizada o precisa del problema. Esta capacidad de adaptación es esencial en las tecnologías de la información, y en especial en la ciberseguridad ya que es un campo que se encuentra en constante evolución. Teniendo esto en cuenta, no solo se ha descubierto el potencial de las redes neuronales para optimizar y mejorar las tecnologías de la información, incluyendo la ciberseguridad de los sistemas, si no que además se ha descubierto el potencial que tienen las redes neuronales para resolver problemas complejos y su alta capacidad de adaptación a cambios en el problema que resuelven.

%Puesto que se han completado todos los objetivos del proyecto, así como los objetivos académicos fijados al comienzo de la memoria, se puede afirmar que el trabajo ha sido un éxito. Se ha desarrollado un modelo capaz de detectar intrusiones en sistemas informáticos y clasificarlas en nueve tipos diferentes de intrusiones. Además, se han obtenido conocimientos acerca de las redes neuronales que se ignoraban por parte del alumno antes del desarrollo de este trabajo.


\section{Trabajo futuro}
En un posible trabajo futuro se podría explorar como afecta el uso de las técnicas que se comentan en esta sección en la eficacia y en la eficiencia de los modelos desarrollados, con el objetivo de obtener modelos que convejan hacia una solución que generalice mejor, especialmente en el modelo de clasificación multiclase.

\begin{itemize}
	\item Modificar la arquitectura del modelo de clasificación multiclase. Durante el desarrollo del trabajo se han comparado los resultados que se obtenían al modificar el tamaño de la capa oculta. Estas comparaciones han demostrado que en función de la complejidad del problema aumentar el número de neuronas puede ser beneficioso o contraproducente. También se ha probado a añadir una tercera capa en el modelo de clasificación multiclase para comprobar si aplicando esta técnica, el modelo conseguía compensar el desbalanceo en el número de muestras de algunas clases del conjunto de datos con la complejidad de la red neuronal. Desafortunadamente, el valor más alto obtenido en la métrica \textit{F1-weighted} utilizando una arquitectura de tres capas, no superó los resultados obtenidos por los modelos con dos capas.
	
	Sin embargo, al modificar la arquitectura del modelo de clasificación multiclase, sería posible lograr que el modelo converja hacia una solución que sea capaz de identificar con una mayor exactitud aquellas clases con un menor número de muestras, mejorando de esta manera la probabilidad de identificar tanto las clases minoritarias como las mayoritarias.
	
	\item Balancear las clases que identifica el modelo de clasificación multiclase añadiendo nuevos datos al conjunto de datos utilizado durante la fase de entrenamiento. Como se ha comentado en varias ocasiones, el conjunto de datos utilizado durante el entrenamiento de los modelos tiene una proporción realista entre los datos de las conexiones que lo conforman. Que el conjunto de datos presente esta característica es fundamental para comprobar cual sería el comportamiento del modelo en un entorno de producción real. El aspecto negativo, es que en la realidad no existe un equilibrio entre los tipos de intrusiones que sufre un sistema informático, por lo que los datos se encuentran extremadamente desbalanceados.
	
	Al utilizar datos balanceados durante la fase de entrenamiento del modelo y datos realistas durante la fase de evaluación del mismo, se obtendría un modelo capaz de distinguir mejor entre las nueve clases de intrusiones que se tratan en este trabajo y se podría comprobar la eficacia real del modelo de clasificación multiclase al evaluarlo con un conjunto de datos que representan de manera más fideligna las intrusiones que puede sufir un sistema informático.
	
	\item Desplegar los modelos para integrarlos en entornos operativos. Tal y como se comenta en el Capítulo \ref{cap.despliegue}, desplegar los modelos contribuiría a mejorar la seguridad en redes mediante la automatización de la detección de conexiones malignas, como sucede con los IDS. Una vez detectada la actividad maliciosa, se podría dar respuestas rápidas a las intrusiones para mitigar sus efectos, como hacen los IPS.
	
	\item Entenar los modelos con los datos que estos reciben una vez desplegados. En las secciones anteriores se comenta que una de las ventajas de utilizar modelos neuronales para la detección de intrusiones en vez de algoritmos, es la capacidad de los modelos de aprender de las conexiones que reciben mientras están en producción y su potencial para detectar patrones no lineales que pueden presentar nuevos tipos de ataques desconocidos como los \textit{zero day}.
	
%	\item Comparar los resultados obtenidos con los resultados de articulos publicaciones que utilicen el mismo \textit{dataset} o traten el tema de manera similar. Contrastar los resultados obtenidos con estudios previos que emplean el mismo dataset o abordan problemáticas similares permite evaluar objetivamente el desempeño del enfoque propuesto. Esta práctica no solo aporta rigor científico al análisis, sino que también facilita la identificación de avances, limitaciones y oportunidades de mejora en relación con investigaciones existentes.
\end{itemize}



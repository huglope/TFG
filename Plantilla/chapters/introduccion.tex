Este documento corresponde con la memoria del Trabajo de Fin de Grado (TFG) del grado en Informática de la Universidad de Valladolid. Este trabajo se centra en la creación de un modelo neuronal capaz de detectar intrusiones en una red informática. La principal ventaja de utilizar un modelo neuronal para la detección de intrusiones en una red, frente a los algoritmos tradicionales (como firmas basadas en reglas o análisis estadísticos), radica en su capacidad para aprender patrones complejos y no lineales en los datos, lo que le permite identificar amenazas desconocidas o variantes de ataques existentes (zero-day attacks). Mientras que los métodos tradicionales dependen de reglas predefinidas y actualizaciones manuales para detectar intrusiones (limitándose a ataques conocidos), las redes neuronales pueden analizan grandes volúmenes de tráfico de red, detectando anomalías sutiles y correlaciones ocultas mediante capas de abstracción.

\section{Explicación del problema} \label{sec.exp-problema}
En la actualidad, los sistemas informáticos reciben muchos más ataques de denegación de servicio y de intrusión que hace unos años, esto se debe en parte a los avances en los modelos de IA.


Los sistemas informáticos enfrentan actualmente graves amenazas debido al uso malintencionado de la Inteligencia Artificial (IA) por parte de ciberdelincuentes. Una de las principales problemáticas es la automatización de ataques, donde herramientas basadas en IA permiten ejecutar campañas de ataques informáticos con mayor precisión y escala. Estas IAs pueden generar mensajes convincentes, imitar patrones de comportamiento legítimos y evadir medidas de seguridad tradicionales, lo que incrementa la frecuencia y sofisticación de los ataques.  

Otro desafío crítico es la explotación de vulnerabilidades mediante IA, que acelera la identificación de fallos en sistemas sin intervención humana. Existen algoritmos de machine learning que analizan grandes volúmenes de datos para descubrir brechas de seguridad en tiempo récord, facilitando ataques dirigidos incluso contra infraestructuras críticas como hospitales. 

La IA también complica la defensa, ya que los sistemas de detección tradicionales no siempre pueden anticipar tácticas adaptativas generadas por algoritmos hostiles. Esto obliga a las organizaciones y empresas a invertir en soluciones de IA defensiva, como sistemas de respuesta autónoma. Sin embargo, esto genera una carrera tecnológica desigual donde actores maliciosos aprovechan herramientas accesibles y de bajo costo. La falta de regulación global agrava este escenario, dificultando la mitigación de riesgos asociados.  

Además, los modelos neuronales son adaptativos: mejoran su precisión con el tiempo al entrenarse con nuevos datos, lo que es crucial en entornos dinámicos donde los ciberataques evolucionan rápidamente. Por ejemplo, pueden distinguir entre comportamientos legítimos inusuales (como un empleado accediendo a recursos fuera de horario) y actividades maliciosas (como filtración de datos), reduciendo falsos positivos. En cambio, los enfoques tradicionales suelen ser rígidos y requieren ajustes manuales frecuentes para mantener su eficacia.

Sin embargo, el uso de modelos nueronales para la defensa de los sistemas conlleva grandes desafíos, como la necesidad de grandes conjuntos de datos etiquetados y recursos computacionales intensivos. Aun así, en escenarios donde la sofisticación de los ataques supera las capacidades de detección convencionales, los modelos neuronales representan un salto cualitativo en proactividad y escalabilidad. 


https://www.wsj.com/articles/the-ai-effect-amazon-sees-nearly-1-billion-cyber-threats-a-day-15434edd


\section{Motivación} \label{sec.motivacion}
A continuación, se explica cual ha sido la motivación para realizar este proyecto. La motivación representa la fuerza impulsora o el conjunto de razones que justifican su inicio y continuidad. La motivación puede originarse de la necesidad de resolver un problema específico, aprovechar una oportunidad identificada, cumplir con requisitos normativos, alcanzar metas estratégicas o generar un impacto positivo

Durante mi formación universitaria en el Grado en Ingeniería Informática, como alumno 
de la mención de tecnologías de la información, he aprendido a administrar grandes sistemas de computación en aspectos como: la seguridad, la garantía de la información, la evaluación de dichos sistemas y el almacenamiento de los datos. Además de cierto componente de desarrollo de software.

\textbf{Revisar}

Sin embargo, uno de los conocimientos que no he podido adquirir durante mis estudios, es uno de los temas más importantes en la actualidad, la Inteligencia Artificial. Con el objetivo de expandir mis conocimientos sobre este tema, decidí implementar un modelo neuronal que facilitase la detección de ataques a redes informáticas que tantas complicaciones está generando a los encargados de la administración de estos sistemas.

 
\section{Objetivos del proyecto} \label{sec.objetivos-pro}
En esta sección se listan los objetivos del proyecto, que constituyen las metas específicas, medibles, alcanzables, relevantes y con plazos definidos que se persiguen con la ejecución del mismo. Dichos objetivos describen los resultados concretos que se espera lograr al finalizar el proyecto y proporcionan un marco de referencia para la planificación, la ejecución, el seguimiento y la evaluación de su progreso.

\begin{itemize}
\item Investigar las mejores opciones de arquitectura y de elección de hiperparámetros.
\item Entendimiento de los problemas que enfrentan los sistemas informáticos en la actualidad
\item Generación de modelos basados en Deep Learning.
\item Mitigar riesgos de seguridad, reduciendo los tiempos de respuesta ante incidentes.

\end{itemize}


\section{Objetivos académicos} \label{sec.objetivos-aca}
\begin{itemize}
\item Aprender acerca del funcionamiento de los modelos neuronales y los diferentes tipos de ellos que existen.
\item Aprender acerca del funcionamiento de los modelos neuronales y los diferentes tipos de ellos que existen.
\end{itemize}


\section{Estrucutra de la memoria} \label{sec.estr-memoria}

Este documento se estructura de la siguiente forma:
\begin{description}
\item[Capítulo 1 Introducción:]
\item[Capítulo 2 Metodología:] 
\item[Capítulo 3 Planificación:]
\item[Capítulo 4 Entendimiento del problema:]
\item[Capítulo 5 Entendimiento de los datos:]
\item[Capítulo 6 Modelos:]
\item[Capítulo 7 Test:]
\item[Capítulo 8 Despliegue:]
\item[Capítulo 9 Tecnologías utilizadas:]
\item[Capítulo 10 Seguimiento del proyecto:]
\item[Capítulo 11 Conclusiones:]
\item[Anexo A Manuales:]
\item[Anexo B Resumen de enlaces adicionales:]
\end{description}

Este documento corresponde con la memoria del Trabajo de Fin de Grado (TFG) del Grado en Ingeniería Informática de la Escuela de Ingeniería Informática (EII) de la Universidad de Valladolid (UVa). Este trabajo se centra en el diseño e implementación de un modelo neuronal capaz de detectar intrusiones en una red informática. El modelo está dividido en dos niveles de clasificación, un modelo de clasificación binaria que clasifica las conexiones como benignas o malignas, y un modelo de clasificación multiclase que recibe como entradas las conexiones clasificadas por el modelo binario como malignas. Este último modelo distingue en nueve tipos diferentes de intrusiones las conexiones que recibe, es decir, trata de determinar el tipo de intrusión. El uso de redes neuronales para la detección de intrusiones permite identificar patrones complejos y no lineales en el tráfico de red, mejorando la capacidad de detectar ataques desconocidos o sofisticados. Además, las redes neuronales pueden adaptarse y aprender continuamente a partir de nuevos datos, aumentando su precisión con el tiempo.

%La principal ventaja de utilizar un modelo neuronal para la detección de intrusiones en una red, frente a los algoritmos tradicionales (como firmas basadas en reglas o análisis estadísticos), radica en su capacidad para aprender patrones complejos y no lineales en los datos, lo que le permite identificar amenazas desconocidas o variantes de ataques existentes (zero-day attacks). Mientras que los métodos tradicionales dependen de reglas predefinidas y actualizaciones manuales para detectar intrusiones (limitándose a ataques conocidos), las redes neuronales pueden analizan grandes volúmenes de tráfico de red, detectando anomalías sutiles y correlaciones ocultas mediante capas de abstracción.

\section{Contexto} \label{sec.exp-problema}
En la actualidad, los sistemas informáticos reciben muchos más ataques de denegación de servicio y de intrusión que hace unos años, esto se debe en parte a los avances en los modelos de Inteligencia Artificial (IA). Los sistemas informáticos enfrentan actualmente graves amenazas debido al uso malintencionado de la IA por parte de ciberdelincuentes. Una de las principales problemáticas es la automatización de ataques, donde herramientas basadas en IA permiten ejecutar campañas de ataques informáticos con mayor precisión y escala. Estas IAs pueden generar mensajes convincentes, imitar patrones de comportamiento legítimos y evadir medidas de seguridad tradicionales, lo que incrementa la frecuencia y sofisticación de los ataques. 

Otro desafío crítico es la explotación de vulnerabilidades mediante IA, que acelera la identificación de fallos en sistemas sin intervención humana. Existen algoritmos de aprendizaje automático (\textit{Machine Learning}, ML) que analizan grandes volúmenes de datos para descubrir brechas de seguridad en tiempo récord, facilitando ataques dirigidos incluso contra infraestructuras críticas como hospitales. Por estos motivos, la ciberseguridad es una tema de actualidad del que cada vez más se habla con más frecuencia incluso en la prensa general o económica \cite{rundle2024ai}.

La IA también complica la defensa, ya que los sistemas de detección tradicionales no siempre pueden anticipar tácticas adaptativas generadas por algoritmos hostiles. Esto obliga a las organizaciones y empresas a invertir en soluciones de IA defensiva, como sistemas de respuesta autónoma. Sin embargo, esto genera una carrera tecnológica desigual donde actores maliciosos aprovechan herramientas accesibles y de bajo costo. La falta de regulación global agrava este escenario, dificultando la mitigación de riesgos asociados.  

Además, los modelos neuronales son adaptativos: mejoran su precisión con el tiempo al entrenarse con nuevos datos, lo que es crucial en entornos dinámicos donde los ciberataques evolucionan rápidamente. Por ejemplo, pueden distinguir entre comportamientos legítimos inusuales (como un empleado accediendo a recursos fuera de horario) y actividades maliciosas (como filtración de datos), reduciendo falsos positivos. En cambio, los enfoques tradicionales suelen ser rígidos y requieren ajustes manuales frecuentes para mantener su eficacia.

Sin embargo, el uso de modelos nueronales para la defensa de los sistemas conlleva grandes desafíos, como la necesidad de grandes conjuntos de datos etiquetados y recursos computacionales intensivos. Aun así, en escenarios donde la sofisticación de los ataques supera las capacidades de detección convencionales, los modelos neuronales representan un salto cualitativo en proactividad y escalabilidad. 



\section{Motivación} \label{sec.motivacion}

La principal motivación, que impulsó el desarrollo de este proyecto es facilitar la detección de ataques en redes informáticas, que tantas complicaciones está generando a los encargados de la administración de estos sistemas. Para complir con esta motivación, se decidió implementar un modelo neuronal que cumpliese con estos requisitos.

Durante la formación universitaria en el Grado en Ingeniería Informática,los alumnos
de la mención de tecnologías de la información, aprenden a administrar grandes sistemas de computación en aspectos como: la seguridad, la garantía de la información, la evaluación de dichos sistemas y el almacenamiento de los datos. Además de contener formación sobre ciertos componentes de desarrollo de software. La falta de formación en algunos aspectos de la IA en dicha mención, desemboca en una de las grandes motivaciones académicas para desarrollar este proyecto, como es la familiarización con las técnicas de ML, en concreto de las redes neuronales, como herramientas útiles.

 
\section{Objetivos del proyecto} \label{sec.objetivos-pro}
En esta sección se listan los objetivos del proyecto, que constituyen las metas específicas, medibles, alcanzables, relevantes y con plazos definidos que se persiguen con la ejecución del mismo. Dichos objetivos describen los resultados concretos que se espera lograr al finalizar el proyecto y proporcionan un marco de referencia para la planificación, la ejecución, el seguimiento y la evaluación de su progreso.

\begin{enumerate}
	\item Diseñar e implementar un modelo capaz de detectar intrusiones en redes informáticas y proporcinar una clasificación previa de la intrusión.
	\item Los modelos de detección han de ser modelos neuronales.
	\item Evaluar y comparar los modelos generados con un dataset real y complejo .
\end{enumerate}

\section{Objetivos académicos} \label{sec.objetivos-aca}
En esta sección se enumeran los objetivos académicos del presente estudio, los cuales representan las metas específicas, susceptibles de evaluación, realizables, pertinentes para el ámbito del conocimiento, que se pretenden alcanzar a través del desarrollo de este proyecto. 

\begin{enumerate}
	\item Comprender el funcionamiento de los modelos neuronales através de  pytorch y las métricas de evaluación.
	\item Aprender las características de varios de los tipo de modelos neruonales que existen.
	\item Descubrir el potencial de las redes neuronales para optimizar y mejorar las tecnologías de la información, incluyendo la ciberseguridad de los sistemas.

\end{enumerate}


\section{Estrucutra de la memoria} \label{sec.estr-memoria}

Este trabajo fin de grado se estructura de la siguiente manera:
\begin{description}
\item[Capítulo 2 Metodología:] En este capítulo se definen cuales son las fases de la metodología CRISP-DM que se utiliza como metodología y modelo de proceso para el diseño y evaluación de los modelos neuronales para la detección de intrusiones. Se describe la aplicación de cada una de las seis fases al contexto específico del desarrollo de modelos neuronales.

\item[Capítulo 3 Planificación:] Este capítulo presenta la planificación del proyecto. Se definen los recursos necesarios, se identifican las tareas principales, se estiman los plazos y se establece el cronograma. También se aborda la gestión de riesgos inicial y la asignación de roles.

\item[Capítulo 4 Entendimiento del problema:] En este capítulo se describe el problema que el proyecto busca abordar. Se presenta el contexto, la relevancia y los objetivos generales.

\item[Capítulo 5 Entendimiento de los datos:] Este capítulo se dedica a la exploración y comprensión del conjunto de datos utilizado para el entrenamiento y evaluación de los modelos desarrollados. Se describe la fuente, el formato, el tamaño y las variables de los datos. Se presenta un análisis exploratorio para identificar patrones, problemas de calidad y la distribución de las variables.

\item[Capítulo 6 Modelos:] En este capítulo se detallan los modelos de redes neuronales desarrollados y entrenados. Se describe la arquitectura, la justificación de su elección, los hiperparámetros, la función de pérdida y el optimizador. Se incluye la estrategia de entrenamiento y las métricas de evaluación.

\item[Capítulo 7 Test:] Este capítulo se centra en la evaluación final de los modelos entrenados. Se describe el conjunto de datos de prueba, el proceso de evaluación, la presentación de los resultados de las métricas y el análisis de las fortalezas y debilidades de los modelos.

\item[Capítulo 8 Despliegue:] Este capítulo aborda la fase de despliegue de los modelos entrenados. Se describe la integración en un entorno operativo, las consideraciones técnicas, los posibles desafíos y las estrategias de monitorización y mantenimiento. Debido a la diversidad de entornos existentes en los que se puede realizar el despliegue, se comentarán los pasos generales que habría que seguir para desplegar los modelos en un entorno de producción real, sin entrar en profundidad.

\item[Capítulo 9 Tecnologías utilizadas:] En este capítulo se listan y describen las tecnologías de software y hardware empleadas en el proyecto. Se incluyen lenguajes de programación, bibliotecas de aprendizaje automático, herramientas de visualización y plataformas de seguimiento de experimentos.

\item[Capítulo 10 Seguimiento del proyecto:] Este capítulo describe cómo se ha realizado el seguimiento del progreso del proyecto. Se definen los indicadores clave de rendimiento, las metodologías de seguimiento, las herramientas de gestión y los mecanismos para la identificación y resolución de desviaciones.

\item[Capítulo 11 Conclusiones:] En este capítulo final se presentan las conclusiones del proyecto. Se resumen los principales hallazgos, se evalúa el cumplimiento de los objetivos académicos, se discuten las implicaciones de los resultados, las limitaciones y las posibles líneas de trabajo futuro.
\end{description}

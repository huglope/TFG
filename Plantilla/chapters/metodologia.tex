
\section{CRISPDM}
La metodología CRISP-DM (Cross-Industry Standard Process for Data Mining) es un marco de trabajo estandarizado para guiar proyectos de minería de datos y aprendizaje automático. Su estructura cíclica y flexible la hace aplicable en diversos dominios, desde marketing hasta ciberseguridad. Está compuesta por estas fases:

1. Entendimiento del problema: Esta fase inicial se centra en alinear los objetivos técnicos con las necesidades del negocio o los problemas a resolver. Implica definir requisitos, identificar métricas de éxito y comprender el contexto organizacional. Por ejemplo, en un proyecto de detección de fraude, se establecerían criterios para medir la eficacia del modelo (como reducir falsos negativos en un 20%). 

2. Entendimiento de los datos: En esta fase se recopilan y exploran los datos disponibles para evaluar su calidad, relevancia y limitaciones. Mediante análisis descriptivo y visualizaciones, se identifican patrones preliminares, valores atípicos o sesgos.   

3. Preparación de los Datos: Fase crítica donde los datos brutos se transforman en un conjunto adecuado para modelado. Incluye limpieza (manejo de valores nulos), integración de fuentes diversas, normalización y creación de características (feature engineering).

4. Modelado: Se seleccionan y entrenan algoritmos (como redes neuronales o árboles de decisión) para resolver el problema definido. La elección del modelo depende de la naturaleza de los datos y los objetivos (clasificación, regresión, etc.). Se aplican técnicas de validación (como k-fold cross-validation) para evitar sobreajuste.  

5. Evaluación: Los modelos se prueban con métricas rigurosas (precisión, recall, AUC-ROC) para determinar si cumplen los criterios de negocio establecidos en la Fase 1. También se valida su robustez en escenarios realistas.

6. Despliegue: Una vez validado, el modelo se implementa en producción, integrado en sistemas operativos. Esto incluye monitorización continua, actualizaciones y documentación para usuarios finales. En agricultura inteligente, un modelo de predicción de cosechas podría desplegarse en una app móvil para que los agricultores reciban alertas en tiempo real.  

CRISP-DM es iterativa, esto significa que los resultados de fases posteriores pueden revelar la necesidad de ajustes en etapas anteriores (como recolectar más datos o redefinir objetivos). Su enfoque estructurado minimiza riesgos y maximiza el valor entregado, siendo especialmente útil en proyectos complejos donde la alineación entre técnica y negocio es esencial.
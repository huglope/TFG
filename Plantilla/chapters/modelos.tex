\section{Métricas}  \label{sec.metricas}
\subsection{Matriz de confusión} \label{sec.matriz-consfusion}
Para evaluar el desempeño del modelo de detección y clasificación de ataques, se utilizan las siguientes métricas derivadas de la matriz de confusión.

\begin{table}[h]
\centering
\label{tab:confusion_matrix}
\begin{tabular}{|l|c|c|}
\hline
 & \textbf{Predicción Positiva} & \textbf{Predicción Negativa} \\ \hline
\textbf{Real Positivo} & Verdaderos Positivos (VP) & Falsos Negativos (FN) \\ \hline
\textbf{Real Negativo} & Falsos Positivos (FP) & Verdaderos Negativos (VN) \\ \hline
\end{tabular}
\caption{Matriz de confusión para clasificación binaria.}
\end{table}

\subsection{Fórmulas e Interpretación}

\begin{itemize}
    \item \textbf{Exactitud (\textit{Accuracy})}: \label{met:Accuracy}
    
\begin{equation}
    \text{Accuracy} = \frac{VP + VN}{VP + FP + VN + FN}
\end{equation}

En el entrenamiento de modelos neuronales para la detección de intrusiones, esta métrica representa la proporción del total de las clasificaciones realizadas correctamente. Indica la capacidad general del modelo para distinguir entre tráfico normal e intrusivo. Si bien ofrece una visión global del rendimiento, su valor disminuye en escenarios donde la cantidad de tráfico normal supera significativamente al tráfico malicioso, ya que el modelo puede obtener una alta exactitud simplemente clasificando la mayoría de las instancias como normales.

\item \textbf{Precisión (\textit{Precision})}: \label{met:Precision}

\begin{equation}
    \text{Precision} = \frac{VP}{VP + FP}
\end{equation}

Esta métrica evalúa la capacidad del modelo neuronal para evitar la identificación incorrecta de tráfico normal como intrusivo. En la detección de intrusiones, una alta precisión es crucial para minimizar las falsas alarmas, las cuales pueden generar una sobrecarga operativa en los equipos de seguridad, requiriendo la revisión de eventos benignos y distrayendo la atención de amenazas reales. Un modelo preciso reduce la fatiga de alertas y permite una respuesta más eficiente a incidentes genuinos.

\item \textbf{Sensibilidad (\textit{Recall})}: \label{met:Recall}

\begin{equation}
    \text{Recall} = \frac{VP}{VP + FN}
\end{equation}

La sensibilidad mide la habilidad del modelo neuronal para detectar todas las instancias de intrusión presentes en el tráfico de red. En el contexto de la seguridad, un alto recall es de suma importancia, ya que implica una menor probabilidad de que ataques reales pasen desapercibidos. Un modelo con baja sensibilidad puede tener consecuencias graves, permitiendo que actividades maliciosas se infiltren y comprometan la integridad y la confidencialidad de los sistemas.

\item \textbf{Puntuación F1 (\textit{F1-Score})}: \label{met:F1-score}

\begin{equation}
    F1 = 2 \times \frac{\text{Precision} \times \text{Recall}}{\text{Precision} + \text{Recall}}
\end{equation}

Esta métrica proporciona una evaluación equilibrada del rendimiento del modelo neuronal al calcular la media armónica entre la precisión y el recall. En la detección de intrusiones, donde a menudo existe un desequilibrio entre el tráfico normal y el malicioso, el F1-score ofrece una métrica más robusta que la exactitud, ya que considera tanto la capacidad de evitar falsas alarmas como la de detectar todas las intrusiones. Un valor alto de F1-score indica un buen compromiso entre ambas capacidades.

\item \textbf{Puntuación F2 (\textit{F2-Score})}: \label{met:F2-score}

\begin{equation}
    F2 = 5 \times \frac{\text{Precision} \times \text{Recall}}{4 \times \text{Precision} + \text{Recall}}
\end{equation}

Esta variante de la puntuación F pondera la sensibilidad más que la precisión. En el ámbito de la detección de intrusiones, el F2-score resulta útil cuando las consecuencias de no detectar un ataque (falso negativo) se consideran significativamente más perjudiciales que generar una falsa alarma (falso positivo). Al asignar un mayor peso al recall, se prioriza la identificación de la mayor cantidad posible de actividades maliciosas, incluso a expensas de un posible aumento en las falsas alertas.
\end{itemize}


\subsection{Aplicación en Seguridad}	\label{sec:apli-met-seg}
En el contexto de detección de intrusiones:
\begin{itemize}
    \item Un recall alto (> 95\%) asegura que pocos ataques pasan desapercibidos.
    \item La precisión debe optimizarse para reducir la carga operativa de analistas (falsos positivos < 10\%).
    \item El F2-Score es preferible al F1 cuando la prioridad es minimizar riesgos de ataques no detectados.
\end{itemize}


\begin{frame}{Contexto}
    \begin{itemize}
    		\item Intrusiones en sistemas informáticos.
        \vspace{10mm}
        \item Evolución de los ataques a las redes informáticas como consecuencia del uso de IA.
        \vspace{10mm}
        \item Defensa antes posibles intrusiones.
    \end{itemize}
\end{frame}


\begin{frame}{Objetivos del proyecto}
    \begin{itemize}
        \item  Diseñar e implementar un modelo capaz de detectar intrusiones en redes informáticas y proporcinar una clasificación previa de la intrusión.
        \vspace{10mm}
        \item Los modelos de detección han de ser modelos neuronales.
        \vspace{10mm}
        \item Evaluar y comparar los modelos generados con un dataset real y complejo .
    \end{itemize}
\end{frame}

\begin{frame}{Objetivos del académicos}
    \begin{itemize}
        \item Comprender el funcionamiento de los modelos neuronales através de PyTorch y las métricas de evaluación.
        \vspace{10mm}
        \item Aprender las características de varios de los tipo de modelos neuronales que existen.
        \vspace{10mm}
        \item Descubrir el potencial de las redes neuronales para optimizar y mejorar las tecnologías de la información, incluyendo la ciberseguridad de los sistemas.
        \end{itemize}
\end{frame}

\begin{frame}{Contexto}
    \begin{itemize}
    		\item Intrusiones en sistemas informáticos.
        \vspace{10mm}
        \item Evolución de los ataques a las redes informáticas como consecuencia del uso de IA.
        \vspace{10mm}
        \item Defensa ante posibles intrusiones.
    \end{itemize}
\end{frame}


\begin{frame}{Objetivos del proyecto}
    \begin{enumerate}
        \item  Diseñar e implementar un modelo capaz de detectar intrusiones en redes informáticas y proporcionar una clasificación previa de la intrusión.
        \vspace{10mm}
        \item Desarrollar modelos de detección que han de ser modelos neuronales.
        \vspace{10mm}
        \item Evaluar y comparar los modelos generados con un \textit{dataset} real y complejo.
    \end{enumerate}
\end{frame}

\begin{frame}{Objetivos del académicos}
    \begin{itemize}
        \item Comprender el funcionamiento de los modelos neuronales a través de PyTorch y las métricas de evaluación.
        \vspace{10mm}
        \item Asimilar las características de varios tipos de modelos neuronales existentes.
        \vspace{10mm}
        \item Descubrir el potencial de las redes neuronales para optimizar y mejorar las tecnologías de la información, incluyendo la ciberseguridad de los sistemas.
        \end{itemize}
\end{frame}
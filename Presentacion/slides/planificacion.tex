\subsection{Planificación}
\begin{frame}{Planificación del proyecto}


\begin{figure}[H]
\centering
\makebox[\linewidth][c]{%  % Centrado mejorado\\

\resizebox{1\textwidth}{0.8\textheight}{  % Ajustamos la altura sin distorsionar las proporciones
\begin{ganttchart}[
    x unit=0.15cm,         % Mantener el tamaño de las unidades en el eje X
    y unit title=0.35cm,   % Mantener la altura de los títulos
    y unit chart=0.35cm,   % Ajustamos la altura de las filas para que no se vea tan comprimido
    hgrid,
    vgrid={*{1}{dotted}},
    title/.style={draw=none},
    title label font=\scriptsize,  % Reducimos el tamaño de la fuente del título
    bar/.style={fill=blue!30, rounded corners=2pt},
    bar height=0.3,         % Ajustamos la altura de las barras sin cambiar demasiado el espacio entre ellas
    group/.style={draw=black, fill=blue!10},
    milestone/.style={fill=red, rounded corners=2pt},
    bar label font=\tiny,           % Reducimos aún más el tamaño de las etiquetas de las barras
    group label font=\footnotesize, % Reducimos el tamaño de las etiquetas de los grupos
    milestone label font=\tiny,     % Reducimos el tamaño de las etiquetas de los hitos
    expand chart=\linewidth  % Ocupa todo el ancho disponible
]{1}{90}
    % Título principal centrado sobre semanas
    \gantttitle{Diagrama de Gantt del Proyecto}{90} \\
    
    % Cabecera de semanas (13 semanas para 90 días)
    \gantttitlelist{1,2,3,4,5,6,7,8,9,10,11,12,13}{7} \\
    
    % Fases y tareas (igual que antes pero ajustadas visualmente)
    \ganttgroup{1. Comprensión Negocio}{1}{10} \\
    \ganttbar{1.1 Definición objetivos}{1}{5} \\
    \ganttbar{1.2 Análisis requisitos}{6}{10} \\
    
    \ganttgroup{2. Comprensión Datos}{11}{20} \\
    \ganttbar{2.1 Recopilación datos}{11}{15} \\
    \ganttbar{2.2 Análisis exploratorio}{16}{20} \\
    
    \ganttgroup{Sprint 1: Preparación}{21}{35} \\
    \ganttbar{3.1 Limpieza datos}{21}{25} \\
    \ganttbar{3.2 Feature engineering}{26}{30} \\
    \ganttbar{3.3 Normalización}{31}{35} \\
    \ganttmilestone{Hito 1}{35} \\
    
    \ganttgroup{Sprint 2: Modelado}{36}{60} \\
    \ganttbar{4.1 Selección algoritmos}{36}{40} \\
    \ganttbar{4.2 Entrenamiento inicial}{41}{50} \\
    \ganttbar{4.3 Ajuste parámetros}{51}{60} \\
    \ganttmilestone{Hito 2}{60} \\
    
    \ganttgroup{Sprint 3: Evaluación}{61}{80} \\
    \ganttbar{5.1 Validación cruzada}{61}{65} \\
    \ganttbar{5.2 Pruebas rendimiento}{66}{70} \\
    \ganttbar{5.3 Análisis resultados}{71}{80} \\
    \ganttmilestone{Hito 3}{80} \\
    
    \ganttgroup{6. Documentación}{81}{90} \\
    \ganttbar{6.1 Redacción memoria}{81}{85} \\
    \ganttbar{6.2 Preparación defensa}{86}{90} \\
    \ganttmilestone{Entrega Final}{90}
\end{ganttchart}
}
}
\caption{Diagrama de Gantt para la planificación del proyecto.}
\label{fig:gantt}
\end{figure}



\end{frame}




\subsection{Costes}
\begin{frame}{Costes de la puesta en marcha del proyecto}

\begin{equation}
	\text{Amortización del Hardware} = 829\,\text{€} \cdot \frac{1}{8} \cdot \frac{3.5}{12} = 30,22\,\text{€} \nonumber
\end{equation}

\begin{table}[H]
\centering

\resizebox{0.9\textwidth}{!}{
\begin{tabular}{|l|l|c|c|c|}
\toprule
Funcionalidad & Software & Coste Mensual & Duración & Coste Total \\
\midrule
Sistema Operativo (SO) & Kubuntu 24.10 x86\_64  & 0\,€  & 3,5 meses & 0\,€ \\
Lenguaje (memoria) & Latex &  0\,€ & 3 meses & 0\,€ \\
Editor latex & TexMaker & 0\,€ & 3 meses & 0\,€ \\
IDE  & MS Visual Studio Code & 0\,€ & 1 mes & 0\,€ \\
Lenguaje (modelos) & Python & 0\,€ & 1 mes & 0\,€ \\
IDE de Python & Jupyeter Notebooks & 0\,€ & 1 mes & 0\,€ \\
Plataforma MLOps & Weights\&Biases & 0\,€ & 1 mes & 0\,€ \\
Control de versiones & GitHub & 0\,€ & 1 mes & 0\,€ \\
IA generativa (código) & DeepSeek & 0\,€ & 1 mes & 0\,€ \\
IA generativa (memoria) & Gemini & 0\,€ & 2 mes & 0\,€ \\
Comunicación 1 & MS Outlook & 0\,€  & 3,5 mes & 0\,€ \\
Comunicación 2 & MS Teams & 0\,€ & 3,5 mes & 0\,€ \\
\bottomrule
\end{tabular}
}
\caption{Costes del Software.}
\label{tab:costes_software}
\end{table}


\end{frame}




\begin{frame}{Costes de la puesta en marcha del proyecto}

\begin{table}[H]
\centering
\begin{tabular}{|l|c|r|r|}
\toprule
\textbf{Profesional} & \textbf{Horas Totales} & \textbf{€/hora} & \textbf{Total (€)} \\
\midrule
Ingeniero ML & 137 & 42,0 & 5\,754,0 \\
Científico de Datos & 113 & 34,5 & 3\,898,5 \\
Analista de Datos & 50 & 18,5 & 925,0 \\
\midrule
\textbf{Costes MOD} & 300 & & \textbf{10\,577,5} \\
\bottomrule
\end{tabular}
\caption{Costes de la mano de obra directa.}
\label{tab:costes_totales}
\end{table}

\end{frame}

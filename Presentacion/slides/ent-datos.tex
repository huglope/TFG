\subsection{Características de los datos}
\begin{frame}{Origen de los datos}

\begin{itemize}
	\item El dataset utilizado en este trabajo es \texttt{NF-UNSW-NB15-v3}, desarrollado como parte de un análisis realizado en la Universidad de Queensland, Australia.
	 \vspace{10mm}
	\item Contiene 53 atributos que describen características del tráfico de red y que permiten clasificar las muestras de tráfico en nueve clases de ataques o como conexiones benignas. 
	 \vspace{10mm}
	\item Algunos de los datos que recoge, son:
\begin{itemize}
	\item La duración de la conexión.
	\item Los bytes enviados y los recibidos.
	\item Versiones de los protocolos utilizados.
\end{itemize}

\end{itemize}
\end{frame}

\begin{frame}{Tipos de ataques registrados en el conjunto de datos}

\begin{table}[H]
\centering
\begin{tabular}{|c|r|} 
\hline
\rowcolor[HTML]{f0f7ff}  
\textbf{Clase} & \textbf{Cantidad} \\ \hline
Benigno & 2\,237\,731 \\ \hline
\textit{Fuzzers} & 33\,816 \\ \hline
\textit{Analysis} & 2\,381 \\ \hline
\textit{Backdoor} & 1\,226 \\ \hline
DoS & 5\,980 \\ \hline
\textit{Exploits} & 42\,748 \\ \hline
\textit{Generic} & 19\,651 \\ \hline
\textit{Reconnaissance} & 17\,074 \\ \hline
\textit{Shellcode} & 4\,659 \\ \hline
\textit{Worms} & 158 \\ \hline
\end{tabular}
\caption{Clasificación de amenazas de seguridad del \textit{dataset} \texttt{NF-UNSW- NB15-v3}.}
\label{tab:attacks-tab}
\end{table}

\end{frame}

\subsection{Preparación de los datos}

\begin{frame}{Atributos}
	Los atributos utilizados para desarrollar los modelos de este trabajo son los 53 atributos del conjunto de datos original, excepto los siguientes que han sido eliminados:
	\begin{itemize}
		\item \texttt{IPV4\_SRC\_ADDR}
		\item \texttt{IPV4\_DST\_ADDR}
		\item \texttt{FLOW\_START\_MILLISECONDS}
		\item \texttt{FLOW\_END\_MILLISECONDS}
	\end{itemize}
\end{frame}

\begin{frame}{Etiquetas}
	\begin{itemize}
	
		\item \texttt{Label}: Indica si se trata de una conexión benigna o maligna.
		 \vspace{10mm}
		\item \texttt{Attack}: Indica el tipo de ataque al que corresponde esa conexión.
	\end{itemize}
\end{frame}

